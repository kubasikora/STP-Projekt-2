\documentclass[a4paper,titlepage,11pt,floatssmall]{mwrep}
\usepackage[left=2.5cm,right=2.5cm,top=2.5cm,bottom=2.5cm]{geometry}
\usepackage[OT1]{fontenc}
\usepackage{polski}
\usepackage[utf8]{inputenc}
\usepackage{amsmath}
\usepackage{amssymb}
\usepackage{graphicx}
\usepackage{mathrsfs}
\usepackage{rotating}
\usepackage{pgfplots}
\usetikzlibrary{pgfplots.groupplots}

\usepackage{siunitx}

\usepackage{float}
\definecolor{szary}{rgb}{0.95,0.95,0.95}
\sisetup{detect-weight,exponent-product=\cdot,output-decimal-marker={,},per-mode=symbol,binary-units=true,range-phrase={-},range-units=single}

\SendSettingsToPgf
\title{\bf Sprawozdanie z projektu nr 2\\ Zadanie nr 2.38  \vskip 0.1cm}
\author{Jakub Sikora \and nr albumu 283418}
\date{\today}
\pgfplotsset{compat=1.15}	
\begin{document}


\makeatletter
\renewcommand{\maketitle}{\begin{titlepage}
		\begin{center}{\LARGE {\bf
					Wydział Elektroniki i Technik Informacyjnych}}\\
			\vspace{0.4cm}
			{\LARGE {\bf Politechnika Warszawska}}\\
			\vspace{0.3cm}
		\end{center}
		\vspace{5cm}
		\begin{center}
			{\bf \LARGE Sterowanie procesami \vskip 0.1cm}
		\end{center}
		\vspace{1cm}
		\begin{center}
			{\bf \LARGE \@title}
		\end{center}
		\vspace{2cm}
		\begin{center}
			{\bf \Large \@author \par}
		\end{center}
		\vspace*{\stretch{6}}
		\begin{center}
			\bf{\large{Warszawa, \@date\vskip 0.1cm}}
		\end{center}
	\end{titlepage}
	}
\makeatother
\maketitle

\tableofcontents


% pierwsza sekcja
\chapter{Model procesu }
Rozważany w ramach projektu proces jest opisany następującą transmitancją:

\begin{equation*}
G(s) = \frac{K_0 e^{-T_0 s}}{(T_1 s + 1)(T_2 s + 1)}
\end{equation*} 
\medskip
gdzie $K_0 = 4,3$, $T_0 = 5$, $T_1 = 1,85$ i  $T_2 = 5,35$.\\
Proces ten jest stabilny asymptotycznie, ponieważ oba jego zera mają ujemną część rzeczywistą.

$$
\left[\begin{array}{c}
s_{01} \\
s_{02} \\
\end{array} \right]
= 
\left[\begin{array}{c}
-0,5405	\\
-0,1869 \\
\end{array} \right]
$$

Oprócz dużych stałych czasowych, obiekt ten zawiera człon opóźniający. Znacząco utrudnia to skuteczną regulację tym obiektem.

\chapter{Zadania projektowe}
\section{Transmitancja dyskretna}
Przed przystąpieniem do projektowania regulatorów, wyznaczyłem transmitancję dyskretną $G(z)$ badanego procesu. Zgodnie z poleceniem przyjąłem okres próbkowania $T_p = 0,5s$. Do wyznaczenia transmitancji dyskretnej posłużyłem się ekstrapolatorem zerowego rzędu według którego transmitancja dyskretna ma następującą postać:

\begin{equation*}
G(z) = \frac{z-1}{z} \mathcal{Z} \bigg\{ \mathcal{L}^{-1} \bigg\{ \frac{G(s)}{s} \bigg\} \bigg\}  
\end{equation*}

W matlabie istnieje polecenie \texttt{c2d} do wyznaczania transmitancji dyskretnej na podstawie transmitancji ciągłej. Polecenie to wykorzystałem w skrypcie o nazwie \texttt{zad1.m}. Za jego pomocą uzyskałem następującą transmitancję dyskretną:

\begin{equation*}
G(z) = z^{-10}\frac{0,04818z^{-1} + 0,04268z^{-2}}{1 - 1,674z^{-1} + 0,6951z^{-2}}
\end{equation*} 

Uzyskana transmitancja ma zera w $z_{01} = 0.9108$ i w $z_{02} = 0.7632$ co jest zgodne z oczekiwaniami. Obiekt opisany transmitancją dyskretną powinien być stabilny tak jak obiekt opisany transmitancją ciągłą. Wszystkie zera $G(z)$ znajdują się w kole jednostkowym, co świadczy o asymptotycznej stabilności procesu.

\begin{figure}[H]
\centering
\includegraphics[width = 0.8\textwidth]{../figures/zad1_cont_disc_comp.pdf}
\caption{Odpowiedzi skokowe transmitancji dyskretnej i ciągłej}
\end{figure}

\newpage

Współczynnik wzmocnienia statycznego transmitancji ciągłej można uzyskać obliczając granicę:
\begin{equation*}
K_{stat_{c}} = \lim_{s\to 0} G(s) 
\end{equation*} 

Analogicznie, współczynnik wzmocnienia statycznego transmitancji dyskretnej uzyskujemy obliczając granicę:
\begin{equation*}
K_{stat_{d}} = \lim_{z\to 1} G(z)
\end{equation*}

Korzystając z otrzymanej transmitancji ciągłej otrzymałem $K_{stat_{c}} = 4.3$, natomiast korzystając z transmitancji dyskretnej obliczonej z matlaba również otrzymałem $K_{stat_{d}} = 4,3$. Wzmocnienia te są toższame z dokładnością do błędów numerycznych. Tożsamość dwóch współczynników była oczekiwana i tylko potwierdza zgodność obu opisów jednego procesu dynamicznego.

\section{Równanie różnicowe}
Znając transmitancję dyskretną, z łatwością można wyznaczyć równanie różnicowe opisujące obiekt postaci:

\begin{equation*}
y[k] = \sum_{i=1}^{n}b_{i}y[k - i] + \sum_{i=1}^{m}c_{i}u[k - i]
\end{equation*}

Odwołując się do definicji transmitancji:
\begin{equation*}
G(z) = \frac{Y(z)}{U(z)} = z^{-10}\frac{0,04818z^{-1} + 0,04268z^{-2}}{1 - 1,674z^{-1} + 0,6951z^{-2}}
\end{equation*}

Mnożąc wyrażenie na krzyż otrzymałem:
\begin{equation*}
z^{-10}(0,04818z^{-1} + 0,04268z^{-2})U(z) =  (1 - 1,674z^{-1} + 0,6951z^{-2})Y(z)
\end{equation*}

Dalej:
\begin{equation*}
0,04818z^{-11}U(z) + 0,04268z^{-12}U(z) = Y(z) - 1,674z^{-1}Y(z) + 0,6951z^{-2}Y(z)
\end{equation*}

Następnym krokiem jest zastosowanie odwrotnej transformaty $\mathcal{Z}$:
\begin{equation*}
0,04818u[k-11] + 0,04268u[k-12] = y[k] - 1,674y[k-1] + 0,6951y[k-2]
\end{equation*}

Ostatecznie po uporządkowaniu wyrazów otrzymałem następujące równanie różnicowe:
\begin{equation*}
y[k] = 1,674y[k-1] - 0,6951y[k-2] + 0,04818u[k-11] + 0,04268u[k-12]  
\end{equation*}

\newpage

\section{Regulator PID}
W ramach następnego zadania wykonałem eksperyment Zieglera-Nicholsa w celu doboru odpowiednich nastawów regulatora PID. Eksperyment ten polega na znalezieniu takiego wzmocnienia regulatora P, który sprawi że obiekt przez niego sterowany znajdzie się na granicy stabilności. Wzmocnienie to nazywamy \emph{wzmocnieniem krytycznym}. Odpowiedzią obiektu znajdującego się na granicy stabilności na skok jednostkowy są niegasnące oscylacje. Poszukiwanym przez nas parametrem jest okres tych oscylacji. Znając okres oscylacji i wzmocnienie krytyczne, można wyznaczyć nastawy regulatora PID korzystając z tabelki.
\medskip
\begin{center}
\begin{tabular}{|c|c|c|c|} \hline
Regulator & $K$ & $T_i$ & $T_d$ \\
\hline \hline
P & $0,5K_u$ & - & -\\
\hline
PI & $0,45K_u$ & $0,83T_u$ & - \\
\hline
PID & $0,6K_u$ & $0,5T_u$ & $0,12T_u$ \\
\hline 
\end{tabular}\\
Tabela 2.1 Nastawy regulatorów P, PI, PID.
\end{center}
\bigskip
Eksperyment Z-N wykonałem za pomocą skryptu \texttt{/scripts/zad3\_{}zn.m}. 
Wzmocnienie krytyczne wyznaczałem iteracyjnie. Jeśli na skok jednostkowy, proces odpowiadał gasnącymi oscylacjami, oznaczało to że aktualne wzmocnienie jest za małe. Na rysunku poniżej znajduje się odpowiedź obiektu z regulatorem typu P z wzmocnieniem $K_p = 0,4$.

\begin{figure}[H]
\centering
\includegraphics[width = 0.80\textwidth]{../figures/zad3_zn_exp_too_small.pdf}
\caption{Za małe wzmocnienie regulatora P - gasnące oscylacje}
\end{figure}

\newpage
Przy zwiększaniu wzmocnienia, istniało ryzyko że możemy przekroczyć granicę stabilności. Objawiało się to rosnącymi oscylacjami.

\begin{figure}[H]
\centering
\includegraphics[width = 0.80\textwidth]{../figures/zad3_zn_exp_too_big.pdf}
\caption{Za duże wzmocnienie regulatora P - rosnące oscylacje}
\end{figure}

Ostatecznie udało mi się dobrać odpowiednie wzmocnienie krytyczne które równe jest $K_u = 0,5125$.

\begin{figure}[H]
\centering
\includegraphics[width = 0.80\textwidth]{../figures/zad3_zn_exp.pdf}
\caption{Wzmocnienie krytyczne - obiekt na granicy stabilności}
\end{figure}

\newpage

Okres oscylacji przebiegu z rysunku 2.4 wynosi 41 próbek. Okres ten znalazłem odczytując ile próbek dzieli poszczególne wierzchołki sinusoidy i mnożąc tą ilość przez czas próbkowania. Ostatecznie $T_u = 20,5s$.

Znając wyniki eksperymentu Zieglera-Nicholsa można w prosty sposób obliczyć nastawy regulatora. Podstawiając do wzorów z tabeli 2.1, uzyskałem:

\begin{center}
	$K_p = 0,3075$ \\
	
	$T_i = 10,25$ \\

	$T_d = 2,46$ \\
\end{center}

Uzyskane nastawy dotyczą ciągłego układu regulacji opisanego wzorem:
\begin{equation*}
u(t) = K_p\bigg(e(t) + \int_{0}^{t}e(\tau)d\tau + \frac{de(t)}{dt}\bigg)
\end{equation*}

Dyskretny układ regulacji PID opisany jest wzorem:
\begin{equation*}
u[k] = u[k-1] + r_1e[k] + r_2e[k-1] + r_3e[k-2]
\end{equation*}

Relację łączącą układ ciągły i dyskretny wyznaczają wzory:
\begin{center}
	$r_1 = K_p\bigg(1 + \frac{T_p}{2T_i} + \frac{T_d}{T_p}\bigg)$ \\
	
	$r_2 = K_p\bigg(\frac{T_p}{2T_i} - \frac{2T_d}{T_p} - 1 \bigg)$\\

	$r_3 = K_p\frac{T_d}{T_p}$ \\
\end{center}

Po podstawieniu do wzorów otrzymałem:
\begin{center}
	$r_1 = 1,8279$ \\
	
	$r_2 = -3,3258$ \\

	$r_3 = 1,5129$ \\
\end{center}

Po uzyskaniu nastawów, przetestowałem ten regulator badając jak zachowa się układ regulacji po pobudzeniu go skokiem jednostkowym.

\begin{figure}[H]
\centering
\includegraphics[width = 0.80\textwidth]{../figures/zad4_pid_zn.pdf}
\caption{Odpowiedź układu z regulatorem uzyskanym za pomocą eksperymentu Z-N}
\end{figure}

\bigskip

Jak widać na rysunku 2.5, w układzie regulacji występują duże oscylacje, dodatkowo występuje bardzo duże przeregulowanie wielkości 70 \% skoku sterowania. Dzieje się tak ponieważ regulatory uzyskane za pomocą tej metody mają tendencję do generowania przebiegów oscylacyjnych. Dodatkowo, proces regulacji jest utrudniony przez opóźnienie występujące w obiekcie. Ważnym elementem dobierania regulatora metodą Zieglera-Nicholsa jest ręczne strojenie nastawów.

Ręcznie dostrojone nastawy regulatora PID:

\begin{center}
	$K_p = 0,19$ \\
	$T_i = 8,4$ \\
	$T_d = 2,13$ \\
\end{center}

W użytecznej wersji dyskretnej:

\begin{center}
	$r_1 = 1,0051$ \\  
	$r_2 = -1,8031$ \\ 
	$r_3 = 0,8094$ \\
\end{center}

\begin{figure}[H]
\centering
\includegraphics[width = 0.90\textwidth]{../figures/zad4_pid_zn_popr.pdf}
\caption{Odpowiedź układu z ręczniem dostrojonym regulatorem}
\end{figure}
\newpage 
Regulator ten bardzo dobrze reaguje na zmiany wartości zadanej. Charakteryzuje się jednak gwałtownymi zmianami sterowania.
\begin{figure}[H]
\centering
\includegraphics[width = 0.90\textwidth]{../figures/zad4_pid_zn_popr_zmienne_y_zad.pdf}
\caption{Odpowiedź układu z ręczniem dostrojonym regulatorem na zmienną wartość zadaną.}
\end{figure}

\section{Regulator DMC}
Drugi rozważany regulator implementuje cyfrowy algorytm regulacji DMC w wersji analitycznej bez ograniczeń. Algorytm ten cechuje się tym że w procesie generacji sterowania, posługuje się modelem obiektu w postaci dyskretnych odpowiedzi skokowych postaci $\{s_1, s_2, s_3, ...\}$. Algorytm DMC jest algorytmem predykcyjnym a więc w każdej iteracji rozwiązywany jest problem optymalizacji błędu wyjścia.

\begin{center}

$\min_{\Delta u} = \bigg( \sum_{i=1}^{N}(y_{k+i|k}^{zad} - y_{k+i|k})^2 + \sum_{i=0}^{N_{u} - 1}\lambda(\Delta u_{k+i|k})^2 \bigg)$

\end{center}

W regulatorze DMC przewidywany jest wektor optymalnych sterowań na $N_u$ chwil w przód. Wartość $N_u$ nazywana jest horyzontem sterowania. Wektor ten obliczany jest na podstawie wartości zadanej, aktualnej wartości wyjścia procesu i zmianach sterowania w ostatnich $D-1$ chwilach dyskretnych. Wartość $D$ nazywana jest horyzontem dynamiki. Jego wartość mówi ile próbek dyskretnych musi minąć aby ustaliła się odpowiedź skokowa sterowanego obiektu działającego poza układem regulacji. Wartość wyjść jest przewidywana na $N$ chwil w przód. Wartość $N$ nazywana jest horyzontem predykcji i mówi o tym na ile chwil w przód przewidywane jest wyjście procesu a co za tym idzie ile próbek jest branych do zadania minimalizacji Ostatnim parametrem jest $\lambda$. Ogranicza on zmienność sygnału sterującego. Im jest on większy tym bardziej łagodne przebiegi ma sygnał sterujący.\\
\newpage

Znając model procesu można z łatwością wyznaczyć prawo regulacji. Zanim jednak to nastąpi należy wyznaczyć dwie macierze związane z modelem. Pierwsza z nich to macierz skróconych odpowiedzi skokowych:

$$
\mathbf{M} =
\left[ \begin{array}{ccccc}
s_1 & 0 & \cdots & 0 & 0\\
s_2 & s_1 & \cdots & 0 & 0\\
\vdots & \vdots & \ddots & \vdots & \vdots  \\
s_N & s_{N-1} & \cdots & s_{N - N_u + 2}& s_{N - N_u + 1}\\
\end{array} \right]
$$

Drugą macierzą związaną z modelem jest macierz predykcji opisana wzorem:

$$
\mathbf{M^{P}} =
\left[ \begin{array}{cccc}
s_2 - s_1 & s_3 - s_2 & \cdots & s_D - s_{D-1}  \\
s_3 - s_1 & s_4 - s_2 & \cdots & s_{D+1} - s_{D-1} \\
\vdots & \vdots & \ddots &  \vdots \\
s_{N+1} - s_1 & s_{N+2} - s_2 & \cdots & s_{N + D - 1} - s_{D-1}\\
\end{array} \right]
$$

Na podstawie macierzy odpowiedzi skokowych należy wyznaczyć macierz $\mathbf{K}$:

$$
\mathbf{K} =
\left[ \begin{array}{cccc}
k_{1,1} & k_{1,2} & \cdots & k_{1,N} \\
s_{2,1} & s_{2,2} & \cdots & k_{2,N} \\
\vdots & \vdots & \ddots & \vdots  \\
k_{N_u,1} & k_{N_u,2} & \cdots & k_{N_u,N}\\
\end{array} \right]
$$

Analityczne prawo regulacji DMC ma postać:

\begin{equation*}
\Delta u_{k|k} = k^{e}(y_k^{zad} - y_k) -\sum_{j=1}^{D-1} k_j^u \Delta u_{k-j},
\end{equation*}
\begin{center}
gdzie $k^{e} = \sum_{j=1}^{N} k_{1,j} $ i $
K_{1}M^{P} =
\left[ \begin{array}{cccc}
k_{1}^u & k_{2}^u & \cdots & k_{D-1}^u \\
\end{array} \right]
$
\end{center}

\bigskip
Cały algorytm regulacji DMC został zaimplementowany w skrypcie \texttt{/scripts/zad4\_{}dmc.m}. Do wyznaczania odpowiedzi skokowej zostało wykorzystane równanie róznicowe, badane w sekcji \texttt{Równanie różnicowe}.


\begin{figure}[H]
\centering
\includegraphics[width = 0.70\textwidth]{../figures/zad4_dmc_l_1_N_80_Nu_80_D_80.pdf}
\caption{Przykładowa odpowiedź regulatora DMC z parametrami $N = 80$, $N_u = 80$, $\lambda = 1$}
\end{figure}


\section{Badanie parametrów regulatora DMC}
\subsection{Horyzont dynamiki}
Pierwszym badanym przeze mnie parametrem jest horyzont dynamiki $D$. Parametr ten wyznacza się na podstawie odpowiedzi skokowej obiektu. Opisuje on ile dyskretnych chwil czasu zajmuje obiektowi do ustabilizowania swojej odpowiedzi skokowej.


\begin{figure}[H]
\centering
\includegraphics[width = 0.80\textwidth]{../figures/zad5_disc_odp_skok.pdf}
\caption{Dyskretna odpowiedź skokowa obiektu}
\end{figure}

Na podstawie rysunku 2.9 określiłem ze odpowiedź skokowa ustala się po 40s od momentu skoku sterowania. Przy czasie próbkowania $T_p = 0,5s$ daje 80 dyskretnych próbek. Dlatego też wyznaczony horyzont dynamiki $D = 80$. W kolejnych eksperymentach założyłem że $N_u = N = D = 80$.
\newpage
\subsection{Horyzont predykcji}
Horyzont predykcji jest parametrem który określa na ile chwil wprzód algorytm ma obliczać wyjście procesu. W ramach badań stopniowo zmniejszałem horyzont predykcji i towarzyszący mu horyzont sterowania.

\begin{figure}[H]
\centering
\includegraphics[width = 0.81\textwidth]{../figures/zad5_dmc_l_1_N_80_Nu_80_D_80.pdf}
\caption{Regulator DMC - $\lambda = 1$, $N = 80$, $N_{u} = 80$}
\end{figure}

\begin{figure}[H]
\centering
\includegraphics[width = 0.81\textwidth]{../figures/zad5_dmc_l_1_N_70_Nu_70_D_80.pdf}
\caption{Regulator DMC - $\lambda = 1$, $N = 70$, $N_{u} = 70$}
\end{figure}

\begin{figure}[H]
\centering
\includegraphics[width = 0.85\textwidth]{../figures/zad5_dmc_l_1_N_60_Nu_60_D_80.pdf}
\caption{Regulator DMC - $\lambda = 1$, $N = 60$, $N_{u} = 60$}
\end{figure}

\begin{figure}[H]
\centering
\includegraphics[width = 0.85\textwidth]{../figures/zad5_dmc_l_1_N_40_Nu_40_D_80.pdf}
\caption{Regulator DMC - $\lambda = 1$, $N = 40$, $N_{u} = 40$}
\end{figure}

\begin{figure}[H]
\centering
\includegraphics[width = 0.85\textwidth]{../figures/zad5_dmc_l_1_N_20_Nu_20_D_80.pdf}
\caption{Regulator DMC - $\lambda = 1$, $N = 20$, $N_{u} = 20$}
\end{figure}

\begin{figure}[H]
\centering
\includegraphics[width = 0.85\textwidth]{../figures/zad5_dmc_l_1_N_15_Nu_15_D_80.pdf}
\caption{Regulator DMC - $\lambda = 1$, $N = 15$, $N_{u} = 15$}
\end{figure}

\begin{figure}[H]
\centering
\includegraphics[width = 0.85\textwidth]{../figures/zad5_dmc_l_1_N_12_Nu_12_D_80.pdf}
\caption{Regulator DMC - $\lambda = 1$, $N = 12$, $N_{u} = 12$}
\end{figure}

\begin{figure}[H]
\centering
\includegraphics[width = 0.85\textwidth]{../figures/zad5_dmc_l_1_N_10_Nu_10_D_80.pdf}
\caption{Regulator DMC - $\lambda = 1$, $N = 10$, $N_{u} = 10$}
\end{figure}

Tak jak widać na rysunkach 2.10 - 2.14, zmniejszanie horyzontu predykcji nie wpływa na jakość regulacji. Dopiero gdy horyzont predykcji przekroczy pewną wartość graniczną (w tym przypadku $N = 20$), jakość regulacji znacząco pogarsza się. Przebiegi zaczynają przyjmować charakter mocno oscylacyjny, pojawiają się duże przeregulowania. W przypadku krytycznym w którym długość horyzontu predykcji jest równa dyskretnemu opóźnieniu obiektu, regulator przestaje pracować. Im dłuższy jest horyzont predykcji tym większe zyski z opytmalizacji sterowania ale też dłuższe są obliczenia algorytmu ponieważ macierze modelu zwiększają swój wymiar. Dlatego w tym przypadku dobrze jest przyjąć wyznaczoną wartość graniczną parametru $N$. Na potrzeby dalszych eksperymentów zdecydowałem się na przyjęcie $N = 30$. 

\subsection{Horyzont sterowania}
Horyzont sterowania $N_u$ opisuje na ile chwil w przód obliczana jest wartość sygnału sterującego. Podobnie jak w przypadku horyzontu predykcji, zdecydowałem się na zbadanie jak ten parametr wpływa na jakość regulacji, stopniowo zmniejsząjąc jego wartość. 

\bigskip
\begin{figure}[H]
\centering
\includegraphics[width = 0.85\textwidth]{../figures/zad5_dmc_l_1_N_30_Nu_30_D_80.pdf}
\caption{Regulator DMC - $\lambda = 1$, $N = 30$, $N_{u} = 25$}
\end{figure}

\newpage

\begin{figure}[H]
\centering
\includegraphics[width = 0.85\textwidth]{../figures/zad5_dmc_l_1_N_30_Nu_25_D_80.pdf}
\caption{Regulator DMC - $\lambda = 1$, $N = 30$, $N_{u} = 25$}
\end{figure}

\begin{figure}[H]
\centering
\includegraphics[width = 0.85\textwidth]{../figures/zad5_dmc_l_1_N_30_Nu_20_D_80.pdf}
\caption{Regulator DMC - $\lambda = 1$, $N = 30$, $N_{u} = 20$}
\end{figure}

\begin{figure}[H]
\centering
\includegraphics[width = 0.85\textwidth]{../figures/zad5_dmc_l_1_N_30_Nu_15_D_80.pdf}
\caption{Regulator DMC - $\lambda = 1$, $N = 30$, $N_{u} = 15$}
\end{figure}

\begin{figure}[H]
\centering
\includegraphics[width = 0.85\textwidth]{../figures/zad5_dmc_l_1_N_30_Nu_10_D_80.pdf}
\caption{Regulator DMC - $\lambda = 1$, $N = 30$, $N_{u} = 10$}
\end{figure}

\begin{figure}[H]
\centering
\includegraphics[width = 0.85\textwidth]{../figures/zad5_dmc_l_1_N_30_Nu_5_D_80.pdf}
\caption{Regulator DMC - $\lambda = 1$, $N = 30$, $N_{u} = 5$}
\end{figure}

\begin{figure}[H]
\centering
\includegraphics[width = 0.85\textwidth]{../figures/zad5_dmc_l_1_N_30_Nu_3_D_80.pdf}
\caption{Regulator DMC - $\lambda = 1$, $N = 30$, $N_{u} = 3$}
\end{figure}

\begin{figure}[H]
\centering
\includegraphics[width = 0.85\textwidth]{../figures/zad5_dmc_l_1_N_30_Nu_2_D_80.pdf}
\caption{Regulator DMC - $\lambda = 1$, $N = 20$, $N_{u} = 2$}
\end{figure}


\begin{figure}[H]
\centering
\includegraphics[width = 0.85\textwidth]{../figures/zad5_dmc_l_1_N_30_Nu_1_D_80.pdf}
\caption{Regulator DMC - $\lambda = 1$, $N = 30$, $N_{u} = 1$}
\end{figure}
\newpage

Jak widać na rysunkach 2.18 - 2.26, zmniejszenie horyzontu sterowania sprawia że przebiegi wyjścia są mocniej tłumione, oscylacje nie są już tak widoczne. Najlepszy rezultat uzyskałem dla $N_u = 1$. W ramach regulacji predykcyjnej do generacji zmiany sterowania brane są pod uwagę przyszłe zmiany sterowania. W przypadku obiektów z opóźnieniem, takie rozwiązanie może generować widoczne oscylacje, jednak poprzez przyjęcie $N_u = 1$ tracimy predykcyjne możliwości regulatora, dlatego też do regulacji tego obiektu zdecydowałem się przyjąć $N_u = 2$. Widoczne oscylacje postaram się zlikwidować przy pomocy parametru $\lambda$

\subsection{Współczynnik członu kary}
W raz ze zmianą parametru $\lambda$ zwanego współczynnikiem wagowym członu kary, zmienia się trajektoria sygnału sterującego. Im większy parametr $\lambda$ tym łagodniejsze są przebiegi.

\begin{figure}[H]
\centering
\includegraphics[width = 0.85\textwidth]{../figures/zad5_dmc_l_1_N_30_Nu_2_D_80.pdf}
\caption{Regulator DMC - $\lambda = 1$, $N = 30$, $N_{u} = 2$}
\end{figure}
\newpage

\begin{figure}[H]
\centering
\includegraphics[width = 0.85\textwidth]{../figures/zad5_dmc_l_10_N_30_Nu_2_D_80.pdf}
\caption{Regulator DMC - $\lambda = 10$, $N = 30$, $N_{u} = 2$}
\end{figure}

\begin{figure}[H]
\centering
\includegraphics[width = 0.85\textwidth]{../figures/zad5_dmc_l_100_N_30_Nu_2_D_80.pdf}
\caption{Regulator DMC - $\lambda = 100$, $N = 30$, $N_{u} = 2$}
\end{figure}

\begin{figure}[H]
\centering
\includegraphics[width = 0.85\textwidth]{../figures/zad5_dmc_l_1000_N_30_Nu_2_D_80.pdf}
\caption{Regulator DMC - $\lambda = 1000$, $N = 30$, $N_{u} = 2$}
\end{figure}

\begin{figure}[H]
\centering
\includegraphics[width = 0.85\textwidth]{../figures/zad5_dmc_l_10000_N_30_Nu_2_D_80.pdf}
\caption{Regulator DMC - $\lambda = 10000$, $N = 30$, $N_{u} = 2$}
\end{figure}

Na podstawie rysunków 2.27 - 2.31 można ocenić że optymalną wartością parametru $\lambda$ jest $\lambda = 1000$. Przebieg sygnału sterującego jest gładki i łagodny, w dodatku sygnał wyjściowy szybko nadąża za wartością zadaną. Za duża wartość tego parametru paraliżuje sygnał sterujący co powoduje że czas regulacji jest bardzo długi.

\begin{figure}[H]
\centering
\includegraphics[width = 0.85\textwidth]{../figures/zad5_dmc_l_0,1_N_30_Nu_2_D_80.pdf}
\caption{Regulator DMC - $\lambda = 0.1$, $N = 30$, $N_{u} = 2$}
\end{figure}

\begin{figure}[H]
\centering
\includegraphics[width = 0.85\textwidth]{../figures/zad5_dmc_l_0,01_N_30_Nu_2_D_80.pdf}
\caption{Regulator DMC - $\lambda = 0.01$, $N = 30$, $N_{u} = 2$}
\end{figure}


\begin{figure}[H]
\centering
\includegraphics[width = 0.85\textwidth]{../figures/zad5_dmc_l_0,001_N_30_Nu_2_D_80.pdf}
\caption{Regulator DMC - $\lambda = 0.001$, $N = 30$, $N_{u} = 2$}
\end{figure}


\begin{figure}[H]
\centering
\includegraphics[width = 0.85\textwidth]{../figures/zad5_dmc_l_0,0001_N_30_Nu_2_D_80.pdf}
\caption{Regulator DMC - $\lambda = 0.0001$, $N = 30$, $N_{u} = 2$}
\end{figure}
\newpage

W przypadku zmniejszania parametru $\lambda$ można dostrzec zmianę charakteru sygnału sterującego. W tym przypadku zmienia się on bardzo gwałtownie, jednak nie niesie to żadnych wymiernych korzyści gdyż czasy regulacji procesu są porównywalne z tym osiągniętym przy $\lambda = 1000$, dlatego też ostatecznie zastosowałbym tą wartość. Dla niej osiągnąłem kompromis pomiędzy szybkością regulacji a żądaną postacią sygnału sterującego.

\section{Porównanie regulatorów}

Zanim porównam oba regulatory, przypomnę ich parametry

\subsubsection{Regulator PID}

\begin{center}
	$K_p = 0,19$ \\
	$T_i = 8,4$ \\
	$T_d = 2,13$ \\
\end{center}

W wersji dyskretnej:

\begin{center}
	$r_1 = 1,0051$ \\  
	$r_2 = -1,8031$ \\ 
	$r_3 = 0,8094$ \\
\end{center}

\begin{figure}[H]
\centering
\includegraphics[width = 0.90\textwidth]{../figures/zad4_pid_zn_popr_zmienne_y_zad.pdf}
\caption{Odpowiedź układu z regulatorem PID na zmienną wartość zadaną.}
\end{figure}
\newpage

\subsubsection{Regulator DMC}
\begin{center}
	$D = 80$ \\
	$N = 30$ \\
	$N_u = 2$ \\
	$\lambda = 1000$\\
\end{center}

\bigskip
\begin{figure}[H]
\centering
\includegraphics[width = 0.90\textwidth]{../figures/zad5_dmc_l_1000_N_30_Nu_2_D_80_3sk.pdf}
\caption{Odpowiedź układu z regulatorem DMC na zmienną wartość zadaną.}
\end{figure}


\newpage

\subsubsection{Porównanie przebiegów}
\begin{figure}[H]
\centering
\includegraphics[width = 0.80\textwidth]{../figures/zad6_y_comp.pdf}
\caption{Porównanie przebiegów sygnałów wyjściowych.}
\end{figure}

\begin{figure}[H]
\centering
\includegraphics[width = 0.80\textwidth]{../figures/zad6_u_comp.pdf}
\caption{Porównanie przebiegów sygnałów sterujących.}
\end{figure}
\newpage

Zgodnie z rysunkiem 2.38, widać że regulator PID działa w tym wypadku szybciej niż jego predykcyjny odpowiednik. Czas regulacji obiektu w układzie regulacji z regulatorem PID wynosi $t_{r_{PID}} = 15s$ a w przypadku regulatora DMC $t_{r_{DMC}} = 22,5s$. Wyjście w układzie regulacji PID jest bardziej strome. Świadczą o tym czasy narastania gdzie $t_{n_{PID}} = 8s$ i $t_{n_{DMC}} = 11s$. W przypadku regulatora PID występuje niewielkie ujemne przeregulowanie na poziomie $1-2\%$ natomiast w regulator DMC nie generuje przeregulowań.\\

\indent Bardzo interesujące są przebiegi sygnału sterującego w obu układach regulacji znajdujące się na rysunku 2.39. W przypadku układu regulacji z regulatorem PID, sygnał sterujący zmienia się w bardzo gwałtowny sposób, generując charakterystyczne piki w momentach gdy wartość zadana zmienia się. W przypadku układu regulacji z regulatorem DMC, sygnał sterujący zmienia się w sposób bardzo łagodny. Jakość regulacji z tym regulatorem można spróbować jeszcze bardziej poprawić stosując zmienną trajektorię zadaną przy obliczaniu wektora sterowań. 

\subsubsection{Obszary stabilności}

W trakcie porównywania obu regulatorów, warto sprawdzić jak wrażliwe są one na zmianę parametrów procesu. W ostatnim eksperymencie sprawdziłem w jakim zakresie może zmieniać się wzmocnienie obiektu w funkcji zmiennego opóźnienia. Eksperyment ten przeprowadziłem najpierw zmieniając opóźnienie obiektu a następnie zwiększałem wzmocnienie tak aby doprowadzić obiekt do granicy stabilności. Dodatkowo, w celu prezentacji ciekawych efektów dotyczących regulatora DMC zmieniłem jego parametry na poniższe: 
\begin{center}
	$D = 80$ \\
	$N = 20$ \\
	$N_u = 1$ \\
	$\lambda = 1$\\
\end{center}

Uzyskane wyniki znajdują się rysunkach poniżej. Kolorem niebieskim zaznaczyłem obszary w których regulatory są niestabilne.

\begin{figure}[H]
\centering
\includegraphics[width = 0.80\textwidth]{../figures/pid6_area.pdf}
\caption{Obszar stabilności regulatora PID - kolor niebieski obszar niestabilny}
\end{figure}

\begin{figure}[H]
\centering
\includegraphics[width = 0.80\textwidth]{../figures/dmc6_area.pdf}
\caption{Obszar stabilności regulatora DMC - kolor niebieski obszar niestabilny}
\end{figure}

Na podstawie rysunku 2.40 można stwierdzić że regulator PID jest wrażliwy na rosnące wzmocnienia. Im bardziej rośnie opóźnienie, tym mniejsze jest maksymalne dopuszczalne wzmocnienie obiektu.\\

W przypadku regulatora DMC, sytuacja jest zupełnie odmienna. Regulator ten jest wrażliwy na spadek nominalnego wzmocnienia obiektu. Dodatkowo, w przypadku zwiększenia opóźnienia do poziomu porównywalnemu z horyzontem predykcji, regulator zupełnie przestaje pracować. Efekt ten można zniwelować poprzez zwiększenie horyzontu predykcji.



\end{document}
