\documentclass[a4paper,titlepage,11pt,floatssmall]{mwrep}
\usepackage[left=2.5cm,right=2.5cm,top=2.5cm,bottom=2.5cm]{geometry}
\usepackage[OT1]{fontenc}
\usepackage{polski}
\usepackage[utf8]{inputenc}
\usepackage{amsmath}
\usepackage{amssymb}
\usepackage{graphicx}
\usepackage{mathrsfs}
\usepackage{rotating}
\usepackage{pgfplots}
\usetikzlibrary{pgfplots.groupplots}

\usepackage{siunitx}

\usepackage{float}
\definecolor{szary}{rgb}{0.95,0.95,0.95}
\sisetup{detect-weight,exponent-product=\cdot,output-decimal-marker={,},per-mode=symbol,binary-units=true,range-phrase={-},range-units=single}

\SendSettingsToPgf
\title{\bf Sprawozdanie z projektu nr 2\\ Zadanie nr 2.38  \vskip 0.1cm}
\author{Jakub Sikora \and nr albumu 283418}
\date{\today}
\pgfplotsset{compat=1.15}	
\begin{document}


\makeatletter
\renewcommand{\maketitle}{\begin{titlepage}
		\begin{center}{\LARGE {\bf
					Wydział Elektroniki i Technik Informacyjnych}}\\
			\vspace{0.4cm}
			{\LARGE {\bf Politechnika Warszawska}}\\
			\vspace{0.3cm}
		\end{center}
		\vspace{5cm}
		\begin{center}
			{\bf \LARGE Sterowanie procesami \vskip 0.1cm}
		\end{center}
		\vspace{1cm}
		\begin{center}
			{\bf \LARGE \@title}
		\end{center}
		\vspace{2cm}
		\begin{center}
			{\bf \Large \@author \par}
		\end{center}
		\vspace*{\stretch{6}}
		\begin{center}
			\bf{\large{Warszawa, \@date\vskip 0.1cm}}
		\end{center}
	\end{titlepage}
	}
\makeatother
\maketitle

\tableofcontents


% pierwsza sekcja
\chapter{Model procesu }
Rozważany w ramach projektu proces jest opisany następującą transmitancją:

\begin{equation*}
G(s) = \frac{K_0 e^{-T_0 s}}{(T_1 s + 1)(T_2 s + 1)}
\end{equation*} 
\medskip
gdzie $K_0 = 4,3$, $T_0 = 5$, $T_1 = 1,85$ i  $T_2 = 5,35$.\\
Proces ten jest stabilny asymptotycznie, ponieważ oba jego zera mają ujemną część rzeczywistą.

$$
\left[\begin{array}{c}
s_{01} \\
s_{02} \\
\end{array} \right]
= 
\left[\begin{array}{c}
-0,5405	\\
-0,1869 \\
\end{array} \right]
$$

Oprócz dużych stałych czasowych, obiekt ten zawiera człon opóźniający. Znacząco utrudnia to skuteczną regulację tym obiektem.

\chapter{Zadania projektowe}
\section{Transmitancja dyskretna}
Przed przystąpieniem do projektowania regulatorów, wyznaczyłem transmitancję dyskretną $G(z)$ badanego procesu. Zgodnie z poleceniem przyjąłem okres próbkowania $T_p = 0,5s$. Do wyznaczenia transmitancji dyskretnej posłużyłem się ekstrapolatorem zerowego rzędu według którego transmitancja dyskretna ma następującą postać:

\begin{equation*}
G(z) = \frac{z-1}{z} \mathcal{Z} \bigg\{ \mathcal{L}^{-1} \bigg\{ \frac{G(s)}{s} \bigg\} \bigg\}  
\end{equation*}

W matlabie istnieje polecenie \texttt{c2d} do wyznaczania transmitancji dyskretnej na podstawie transmitancji ciągłej. Polecenie to wykorzystałem w skrypcie o nazwie \texttt{zad1.m}. Za jego pomocą uzyskałem następującą transmitancję dyskretną:

\begin{equation*}
G(z) = z^{-10}\frac{0,04818z^{-1} + 0,04268z^{-2}}{1 - 1,674z^{-1} + 0,6951z^{-2}}
\end{equation*} 

Uzyskana transmitancja ma zera w $z_{01} = 0.9108$ i w $z_{02} = 0.7632$ co jest zgodne z oczekiwaniami. Obiekt opisany transmitancją dyskretną powinien być stabilny tak jak obiekt opisany transmitancją ciągłą. Wszystkie zera $G(z)$ znajdują się w kole jednostkowym, co świadczy o asymptotycznej stabilności procesu.

\begin{figure}[H]
\centering
\includegraphics[width = 0.8\textwidth]{../figures/zad1_cont_disc_comp.pdf}
\caption{Odpowiedzi skokowe transmitancji dyskretnej i ciągłej}
\end{figure}

\newpage

Współczynnik wzmocnienia statycznego transmitancji ciągłej można uzyskać obliczając granicę:
\begin{equation*}
K_{stat_{c}} = \lim_{s\to 0} G(s) 
\end{equation*} 

Analogicznie, współczynnik wzmocnienia statycznego transmitancji dyskretnej uzyskujemy obliczając granicę:
\begin{equation*}
K_{stat_{d}} = \lim_{z\to 1} G(z)
\end{equation*}

Korzystając z otrzymanej transmitancji ciągłej otrzymałem $K_{stat_{c}} = 4.3$, natomiast korzystając z transmitancji dyskretnej obliczonej z matlaba również otrzymałem $K_{stat_{d}} = 4,3$. Wzmocnienia te są toższame z dokładnością do błędów numerycznych. Tożsamość dwóch współczynników była oczekiwana i tylko potwierdza zgodność obu opisów jednego procesu dynamicznego.

\section{Równanie różnicowe}
Znając transmitancję dyskretną, z łatwością można wyznaczyć równanie różnicowe opisujące obiekt postaci:

\begin{equation*}
y[k] = \sum_{i=1}^{n}b_{i}y[k - i] + \sum_{i=1}^{m}c_{i}u[k - i]
\end{equation*}

Odwołując się do definicji transmitancji:
\begin{equation*}
G(z) = \frac{Y(z)}{U(z)} = z^{-10}\frac{0,04818z^{-1} + 0,04268z^{-2}}{1 - 1,674z^{-1} + 0,6951z^{-2}}
\end{equation*}

Mnożąc wyrażenie na krzyż otrzymałem:
\begin{equation*}
z^{-10}(0,04818z^{-1} + 0,04268z^{-2})U(z) =  (1 - 1,674z^{-1} + 0,6951z^{-2})Y(z)
\end{equation*}

Dalej:
\begin{equation*}
0,04818z^{-11}U(z) + 0,04268z^{-12}U(z) = Y(z) - 1,674z^{-1}Y(z) + 0,6951z^{-2}Y(z)
\end{equation*}

Następnym krokiem jest zastosowanie odwrotnej transformaty $\mathcal{Z}$:
\begin{equation*}
0,04818u[k-11] + 0,04268u[k-12] = y[k] - 1,674y[k-1] + 0,6951y[k-2]
\end{equation*}

Ostatecznie po uporządkowaniu wyrazów otrzymałem następujące równanie różnicowe:
\begin{equation*}
y[k] = 1,674y[k-1] - 0,6951y[k-2] + 0,04818u[k-11] + 0,04268u[k-12]  
\end{equation*}

\newpage

\section{Regulator PID}
W ramach następnego zadania wykonałem eksperyment Zieglera-Nicholsa w celu doboru odpowiednich nastawów regulatora PID.





  

\end{document}
